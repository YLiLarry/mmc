\documentclass[acmtoms,acmnow]{acmtrans2m} 
\usepackage{amsmath,amsfonts,amssymb,bbm}
\usepackage{graphicx,url} 
\usepackage{color}
\usepackage[ruled,vlined]{algorithm2e}
        
%% Our definition
\def\K{\mathbb{K}} \def\N{\mathbb{N}} \def\Z{\mathbb{Z}} \def\F{\mathbb{F}}
\def\M{\mathsf{M}} \def\I{\mathsf{I}} \def\R{\mathsf{R}} \def\Q{\mathbb{Q}}
\def\Mat{\mathcal{M}}
%\def\bigO{{\ensuremath{\operatorname{O}}}} 
\def\bigO{{\ensuremath{\mathcal{O}}}}
%\markboth{}{}

%% TeXmacs macros
\newcommand{\tmop}[1]{\ensuremath{\operatorname{#1}}}
\newcommand{\assign}{:=} 
\newcommand{\rem}{\tmop{rem}}

\title{
Simultaneous conversions with the Residue Number System using linear algebra \\
Application to integer matrix multiplication over integers or polynomials
% Integer Matrix Multiplication using Chinese Remainder Theorem
}
            
\author{...
 % Pascal Giorgi - LIRMM CNRS - University of Montpellier \\
 % Romain Lebreton - LIRMM CNRS - University of Montpellier \\
}
           
\begin{abstract} 
  Matrix multiplication with exact results (i.e. integers, finite fields,
  polynomials) is important in computer algebra.  Since today's processors do
  not provide such structures, careful implementation must be done to get high
  performance. 

  In this paper, we focus on matrix multiplication over integers and polynomials
  over a finite field.  We will discuss the use of CRT and Kronecker
  substitution to reduce to multiplication with small entries.

  Our contribution is an algorithm that reduces most of the computations of a
  conversion with the Residue Number System to linear algebra. This allows us to
  benefit from the peak performance of linear algebra implementations to speed
  up simultaneous conversions with the Residue Number System on a wide range of
  input sizes. Note that we do not improve the best asymptotic complexity of
  conversions but dramatically improve the constant of a sub-optimal algorithm.

  Our main application is matrix multiplication over integers and polynomials
  over a finite field. Our speed-up of the conversions to and from the Residue
  Number System improves significantly the overall time of matrix
  multiplication. 
\end{abstract}
      
\category{G.4}{Mathematical Software}{Algorithm design and analysis} 
\category{F.2.1}{Analysis of Algorithms and Problem Complexity}{Numerical Algorithms and Problems}[computations in finite fields.]
\terms{Algorithms, Experimentation, Performance.}
\keywords{}
            
\begin{document}
            
% \begin{bottomstuff}
% \end{bottomstuff}
            
\maketitle

\section{Introduction}
biblio ? checker Bernstein - quel papier ?

Talk about mixed radices litterature ? Who discovered quasi-linear conversions to RNS ?

Importance of matrix multiplication in computer algebra as a source of
implementations at peak performances.

We will use the bit complexity model throughout this paper. The integer $\beta$
will represent the bit-length of a machine word in the complexities. We will
attribute a unit cost to operations such as addition, subtraction,
multiplication, quotient and remainder between single precision integers.

In this paper, we are interested in the latter kind of computation, in the context of multiple
precision arithmetic: we work with matrices or polynomials with coefficients that are
multi-precision integers, or lie in a finite ring $\Z/N\Z$, for some large $N$, and we consider a
basic operation such as the multiplication of these matrices or polynomials. There exist multiple
applications to this fundamental operation; we illustrate this in the last section of this paper
with a discussion of polynomial factorization.

To perform a matrix or polynomial multiplication in such a context, several possibilities exist. A
first approach consists in applying known algorithms, such as Strassen's, Karatsuba's, \dots
directly over our coefficient ring, relying {\it in fine} on fast algorithms for multiplication of
multi-precision integers. Another large class of algorithms relies on {\em modular techniques}, or
{\em residue number systems}, computing the required result modulo many small primes before
recovering the output by Chinese Remaindering.  One should not expect either of these approaches to
be superior in all circumstances. For instance, in extreme cases such as the product of matrices of
size $1$ or $2$ with large integer entries, the modular approach highlighted above is most likely
not competitive with a direct implementation. On the other hand, for the product of larger matrices
or polynomials, residue number systems often perform better than direct implementations, and as
such, they are used in libraries or systems such as NTL, FFLAS-FFPACK, Magma, \dots

In this paper, we present new techniques for residue number systems that are applicable in a wide
range of situations. In many cases, the bottlenecks in such an approach is the reduction of the
inputs modulo many small primes, and the reconstruction of the output from its modular images by
means of Chinese Remaindering; by contrast, operations modulo the small primes are often quite
efficient.

Algorithms of quasi-linear complexity have been known for long for both modular reduction and
Chinese Remaindering~\cite[Chapter~10]{GaGe13}, based on so-called subproduct tree techniques;
however, their practical performance remains somewhat lagging. Our algorithm offers an alternative
to this approach, for those cases where we have several coefficients to convert; it relies on
matrix multiplication to perform these tasks, with matrices that are integer analogues of
Vandermonde matrices and their inverses. As a result, while its complexity is inferior to that of
asymptotically fast methods, our algorithm behaves extremely well in practice, as it allows us to
rely on high-performance libraries for matrix multiplication.

\paragraph{Notations}

%$\M (n)$ polynomial arithmetic and $\I (n)$
Let $\M (n)$ denotes the bit complexity of the multiplication of two $n$-bits integers.
In order to provide a fair etimate of practical complexities, we will consider that we are able to
perform any arithmetic operations on $t$-bits integers at a constant cost. Therefore, let $n=st$ we will consider that the complexity of multiplying $n$-bits integer is $\M(s)\M(t)$ and thus $\M(s)$.

We denote resp. by $(a \rem p)$ and $(a \tmop{quo} p)$ the
remainder and quotient of the Euclidean division of $a \in \mathbbm{Z}$ by
$p \in \N$ where $0 \leqslant (a \rem b) < b$.  We extend the
notation $(a \rem p)$ to any rational $a=n/d \in \Q$ such that
$\gcd (d,p) = 1$ to be the unique residue of $(a \bmod p)$ satisfying
$0 \leqslant (a \rem b) < b$.

For any $a = n / d$ with $d$ coprime to $p$, we let $[a]_p$ be the
unique representative in $\{ 0, \ldots, p - 1 \}$ of $a$ modulo $p$.

We call informally a pseudo-reduction of $a$ modulo $p$ the computation
of $b$ such that $a = b \bmod p$ and $b$ ``not too big'' compared to $p$. In
practice, we often will have that $b =\bigO (p^2)$.

Say that our complexity model is bit complexity.

Should we take $\beta$ (\emph{e.g.} $2^{32}$ or $2^{64}$) as a constant and simplify complexity ?
Yes and no : we will give complexities in
$\bigO( \dots\I(\log \beta) )$ and then specify that the term $\I(\log \beta))$
can be removed when $\beta$ is a constant.

Only give costs for the typical case $p_i \simeq \beta, s \simeq B, r
\gg s$ because of $\beta^B < p_1 \cdots p_s$, which implies that $B < s$ ?

Assume that $n,s \leqslant \beta$ !  Indeed we can assume that
we won't have integers of more than $\beta$ machine words since it
is the typical memory limit for a system whose address are encoded
on a single precision integer. 

We make this restriction to ensure the number of primes $s$
necessary to encode the product of two $N \times N$ matrices of
integers of size $n$ is still $\Theta(n)$ (the equation is
$N \beta^{2n} \leqslant \beta^s$).

Introduce $\M\M(n,k,m)$ as the cost of unbalanced matrix multiplication.

\paragraph{Bibliography} Bibliography on reductions and pseudo-reductions.
Recall Barett, Montgomery results ?

\paragraph{Complexity results}

Let $a \in\Z$ and $p \in \N$ such that $|a|<2^n$ and $p<2^m$.
Cost of $a \tmop{mod} p$ when :

\begin{enumerate}
  \item $n = \Theta (m)$. **Cas classique, on en a vraiment besoin
    -- 
    \begin{equation}\label{eq:complexityReduction1} 
      \bigO \left( \M (m) \right)
    \end{equation}
  
  \item $n \gg \Theta (m)$ **Servira {\`a} prouver l'algo na{\"i}f
    de multi-r{\'e}duction -- 
    \begin{equation}\label{eq:complexityReduction2} 
      \bigO \left( \frac{n}{m} \M(m) \right)
    \end{equation}
  
  \item $n-m \ll \Theta (m)$ **Sert {\`a} montrer que la
    finalisation des pseudos-r{\'e}ductions est peu couteuse**
  
  **Polynomial analog suggests 
\begin{equation}\label{eq:complexityReduction3} 
\bigO \left( \M (n-m) \frac{m}{n-m}
  \right)
\end{equation}

\end{enumerate}



\section{Conversions with Residue Number System}
\label{sec:ConvRNS}

A classic way to represent a positive integer is to use a positional number
system according to a base $\beta$.  Indeed, any integer $a\in\N$ can be encoded
as $(a_{n-1},\hdots,a_1,a_0)_\beta$ since it is uniquely determined by
$a=\sum_{i=0}^{n-1} a_i\beta^i$ assuming $0\leq a_i < \beta$. One advantage of
such a representation is that it allows to represent infinitely many integers
and compare or add/subtract integers in $\bigO(n)$ operations on the digits,
where $n$ is the number of digits of the operands.

However, multiplication with this representation is more complex and requires
$\bigO(n^2)$ operations on the digits. Many other number systems allow to
improve the complexity of the multiplication but most of them loose the benefit
to represent infinitely many integers.

In this article, we do not intend to present all possible representations and
let the reader refer to \cite{BrZi10,Bernstein08} for a survey on fast
integer arithmetic.  Instead, we only present the multi-modular representation
that will allow linear complexity for addition and multiplication of integers
and then describe our method to multiply integer matrices.

\subsection{Residue Number System}

The Residue Number System is non-positional number system that allow to
represent a finite subset of integers.  Let
$M=m_1\times m_2 \times \hdots \times m_s \in \N$ where $m_i\in\N$ such that
$\forall i\neq j,\, m_i\perp m_j$, then any integers $a\in [0,\hdots,M-1]$ can
be uniquely determined by its residue
$( [a]_{m_1}, [a]_{m_2}, \dots, [a]_{m_s})$. The uniqueness of this residual
representation is ensured by the the ring isomorphism
\begin{equation}\label{eq:crt} 
\Z_M \simeq \Z_{m_1}\times \dots \times \Z_{m_s}
\end{equation} 
when the $m_i$'s are coprime. This ring isomorphism is often
called the Chinese Remainder Theorem (CRT)\cite[Section 5.4]{GaGe03}.
 
This residual representation, also called Residue Number System (RNS), provides
groups isomorphism for addition and multiplication which allow to perform these
operations individually on each residual. If the $m_i$ are fixed, let us note
$[a]_i \assign [a]_{m_i}$ so that $([a]_1,[a]_2,\dots,[a]_s)_{RNS}$ is the
representation of the integer $a \in \N$.\\

Let $a,b,c,d \in \N$ such that $c\equiv a + b \bmod M$ and $d \equiv a \times b
\bmod M$ then
\begin{eqnarray}
([c]_1,[c]_2,\dots,[c]_s)_{RNS} &=& ([a]_1+[b]_1 \bmod m_1,\dots, [a]_s+[b]_s \bmod
  m_s)_{RNS} \label{addrns}\\ ([d]_1,[d]_2,\dots,[d]_s)_{RNS} &=& ([a]_1\times [b]_1 \bmod
  m_1,\dots, [a]_s\times [b]_s \bmod m_s)_{RNS} \label{mulrns}
\end{eqnarray}
It is obvious from equations \ref{addrns} and \ref{mulrns} that addition and
multiplication in RNS require $\bigO(s)$ operations on the residuals. One may
note that division in RNS is possible in the same way (individually on each
residual) only when the division is exact.

In order to benefit from RNS representation, one often need to convert back and
forth between classic positional number system and RNS. Of course, these
conversions are costly and must be avoided when possible. However, when the
number of operations in RNS is high enough, these conversions can be neglected
and the RNS approach yields the most efficient solution.

\subsection{Naive approach}

\subsubsection{Conversion to RNS}\label{sssec:naivetoRNS}
In order to convert an integer $a$ to a residue number system
$(m_1,\hdots m_s)$, one can of course apply Euclidean division of $a$ by $m_i$
for $i\in \{1,\hdots,s\}$. In this section, we want to reduce an integer
$a\in\Z$ modulo each $m_1,\dots,m_s\in\N$. Let us assume the particular case
when $m_1,\dots,m_s<\beta$ and $a<\beta^n$, where $n=\Theta(s)$ and $\beta=2^t$ with $t$ related to machine wordsize. 
Indeed, this case is representative of computations that takes advantage of RNS arithmetic on nowadays computers.
Remark: if $n\gg\Theta(s)$, it suffices to compute $\tilde{a}=a \bmod M$ where $M=\Pi_{i=1}^s m_i$ to get back to the case $n=\Theta(s)$. This will add $\bigO(\frac{n}{s}\M(s)\M(t))$ bit
operations to the complexity.
\smallskip

\begin{algorithm}[H]
%\DontPrintSemicolon  
\KwIn{$a=\sum_{i=0}^{n-1}a_i\beta^i, m_j \in \N$} \KwOut{$a \bmod m_j$}
$c=a_{n-1}$\\ \For{$i=n-2$ \textbf{to} $0$} { $r=c\beta \bmod
  m_j$\\ $c=r+a_i$\\ } \Return{$c \bmod m_j$}\;
\caption{Modular reduction}\label{alg:modred}
\end{algorithm}
 
\smallskip
Algorithm \ref{alg:modred} presents the naive approach to perform a modular reduction.
The bit complexity for computing $a \bmod m_j$ is $\bigO(n \M(t))$.
Therefore, conversion to a RNS basis with $s$ moduli costs $\bigO(sn\M(t))$ bit operations.
When $t$ is a constant the cost becomes $\bigO(sn)$.\\ 


The case when the $m_j$'s are almost of the same size as $a$,
i.e. $m_1,\dots,m_s \in \beta^{\Theta(n)}$, is more classical and can be handled
through integer multiplications using Barrett or Montgomery algorithms
\cite{Barrett86,Montgomery85}. Then, the cost of modular reduction becomes
$\bigO(\M(nt)$ bit operations and the cost of conversion to RNS is
$\bigO(s\M(nt))$.  This boils down to $\bigO(s\M(n))$ when $t$ is
fixed.

\subsubsection{Conversion from RNS}

Assuming that an integer $a$ is given by its RNS representation
$([a]_1, \dots, [a]_s)$ with the base $(m_1,\dots,m_s)$. In order to retrieve
the value of $a$ written in base $\beta$, i.e. $a=\sum_{i=0}^{n-1}a_i\beta^i$,
one need to solve the following congruence equations in the $a_i$'s:

\begin{eqnarray}
 \left[a\right]_1 &\equiv& \sum_{i=0}^{n-1}a_i\beta^i \bmod m_1\nonumber \\ 
 & \vdots & \\ 
 \left[a\right]_s & \equiv & \sum_{i=0}^{n-1}a_i\beta^i \bmod m_s \nonumber
\end{eqnarray}
A solution to this system of congruence is given by the Chinese Remainder
Theorem recalled in Equation \ref{eq:crt}.  Let us denote $M=\Pi_{i=1}^sm_i$ and
$M_i=M/m_i$. The solution can be computed using the following
equation
\begin{equation}\label{eq:cra}
a \equiv \sum_{i=1}^s M_i ([a]_iM_i^{-1} \bmod m_i) \bmod M 
\end{equation}

Let us assume, without loss of genericity, that $M$, the $M_i$ and the $M_i^{-1} \bmod m_i$ are precomputed.  
Recalling that $m_i <\beta=2^t$, then each multiplication $a'_i \assign ([a]_i M_i^{-1} \bmod m_i)$ costs
$\bigO(\M(t))$ and each unbalanced multiplication $M_i a'_i$ costs
$\bigO(s \M(t))$.
Since $a'_i < m_i$, by definition of $M_i$ we have $M_ia'_i<M<\beta^s$.
This implies the following bound on the sum 
\[
\sum_{i=1}^s M_i a'_i < s\beta^s
\]
and its reduction cost is $\bigO(t s \log s)$ (using formula
\eqref{eq:complexityReduction3}), which is not dominant. Therefore the total
cost is $\bigO(s^2\M(t))$ bit operations.\\

Concerning the precomputation costs, $M$ can be computed with $s$ unbalanced multiplications at a cost of $\bigO(s^2 \M(t))$ bit operations. Then each unbalanced exact division $M/m_i$ has bit complexity $\bigO(s \M(t))$,
and its corresponding reduction modulo $m_i$ also costs
$\bigO( s \M(t))$ using formula
\eqref{eq:complexityReduction2}. Finally, computing  each $M_i^{-1} \bmod m_i$ cost $\bigO(\M(t)\log t)$. Altogether, the precomputation costs
$\bigO( s^2 \M(t))$ bit operations.

\subsection{Quasi-linear approach}

In this section, we briefly recall the quasi-linear algorithm based on binary
multiplication tree to convert to RNS. We follow the exposition given in 
\cite{GaGe03}[Section 10.3].

\subsubsection{Conversion to RNS}\label{sec:tornsfast}

The first step of the algorithm is a computation of the binary multiplication
tree of the moduli $m_i$. Assume for the sake of simplicity that $s= 2^\kappa$.
Starting from $m_1,\dots,m_s$, we compute the first level of the tree made of
the products $m_1 m_2, m_3 m_4, \dots, m_{s-1} m_s$, then the second level made
of the products $m_1 m_2 m_3 m_4, \dots, m_{s-3} \cdots m_s$ from the first level. And
so on up to the last level of height $\kappa$ made of the full product
$M=m_1 m_2 \cdots m_s$.

The cost of computing this binary multiplication tree is $\bigO(\M(s) \log s~\M(t))$ bit operations.
This boils down to $\bigO(\M(s) \log s)$ when $t$ is a constant.

\medskip

The second step of the algorithm is recursive and use a divide-and-conquer
scheme. In order to reduce an integer $a$ modulo $m_1,\dots,m_s$, one need to compute
\[a_l = a \bmod (m_1 \cdots m_{s/2}) \mbox{ and } a_h = a \bmod (m_{s/2+1} \cdots m_s).\]
 Then recursively reduce $a_l$ modulo $m_1,\dots,m_{s/2}$ and $a_h$ modulo $m_{s/2+1},\dots,m_s$ and you are
done. Note that the products $m_1 \cdots m_{s/2}$ and $m_{s/2+1} \cdots m_s$
were precomputed, and this holds for every products required in the recursive calls.

The cost $C(s)$ of the reduction then satisfies
$C(s) = 2 C(s/2) + \bigO(\M(s) \M(t))$ which yields
$C(s) = \bigO( \M(s) \log s~\M(t))$ as before.

Therefore, the complexity of this approach for conversion to RNS is $\bigO( \M(s) \log s~\M(t))$ bit operations and this becomes $\bigO( \M(s) \log s)$ when $t$ is constant.

\subsubsection{Conversion from RNS}

The backwards conversion is a bit trickier. It still follows Formula
\eqref{eq:cra} together with a divide-and-conquer approach.

The first step is to precompute $M_i^{-1} \bmod m_i$ for all
$1 \leqslant i \leqslant s$. This can be achieved in time $\bigO( \M(s) \log s~\M(t))$. For
this matter, it suffices to use the quasi-linear conversion to RNS of previous section to
compute $M$ modulo $m_1^2,\dots,m_s^2$. Then one can recover
$M_i \bmod m_i = \left(M \bmod m_i^2 \right)/m_i$ with an exact integer
division. It remains to perform the inversions of $(M_i \mod m_i)$ modulo $m_i$
which amounts to $\bigO(s \M(t)\log t)$ bit operations.

Now let $a'_i \assign [a]_iM_i^{-1} \bmod m_i$.  Formula
\eqref{eq:cra} gives $a \equiv \sum_{i=1}^s a'_i (M/m_i) \bmod M$ which can be
computed recursively on the number $s$ of moduli as follow.

Let $M_l \assign m_1 \cdots m_{s/2}$ and $M_h \assign m_{s/2+1} \cdots m_s$.
Compute recursively 
\begin{eqnarray*}
a_l & \assign & \sum_{i=1}^{s/2} a'_i (M_l/m_i) \bmod M\\
a_h & \assign & \sum_{i=s/2+1}^{s} a'_i (M_h/m_i) \bmod M
\end{eqnarray*}
and finally recover $a$ using $a = M_h \cdot a_l + M_l \cdot a_h$. \\
This recursive algorithm costs $C(s) = 2 C(s/2) + \bigO(\M(s) \M(t))$
which yields $C(s) = \bigO( \M(s) \log s~\M(t))$ as before.


\subsection{Summary of complexities}

In the following table, we recall all the complexities for conversions with RNS representation. We assume that the RNS basis is given as $(m_1,\dots,m_s)$ such that each $m_i<2^t$ and integers to
convert have $n$-bits with $n=\Theta(st)$.

\begin{table}[h]
\centering
\begin{tabular}{|l|r|r|}
\hline
               & conversion from RNS & conversion to RNS\\
\hline
naive approach & $\bigO(s^2\M(t))$ & $\bigO(s^2\M(t))$\\
fast approach  & $\bigO(\M(s)\log s~\M(t))$ & $\bigO(\M(s)\log s~\M(t))$ \\
\hline
\end{tabular}
\end{table}

\section{Simultaneous RNS conversions}

In this section, we want to simultaneously reduce the integers
$a_1, \ldots, a_r \in \Z$ modulo each of the positive coprime integers
$m_1, \ldots, m_s \in \N$.
As before, we assume that $m_1, \ldots, m_s < 2^t$ and that $a_j < 2^{nt}$ with $n=\Theta(s)$. In order to simplify the notations and the complexity estimate, we will use $\beta=2^t$ and $n=s$ throughout this section.

\subsection{Straightforward}

The previous algorithms of Section~\ref{sec:ConvRNS} directly apply to
simultaneous conversions with RNS. Of course, the precomputations of these
algorithms are independent of the integer $a$ to reduce. Therefore, they are
done once and for all. The bit complexity of simultaneous conversions with RNS is
$\bigO \left( rs^2 \M (t)\right)$ for the naive approach and
$\bigO \left( r \M(s) \log s \M(t)\right)$ for the quasi-linear
approach.

Note that, in term of implementation, simultaneous reductions using the naive
algorithms can benefit from vectorized (SIMD) instructions to lower the constant
and can be easily parallelized. On the other hand, simultaneous reductions using
the quasi-linear approach do not benefit from SIMD instructions (or at least not
straightforwardly).

\subsection{Linear Algebra}
When one convert an integer $a$ given in base $\beta$ (i.e. $a=\sum_{i=0}^{s-1} a_i\beta^i$, ) to its RNS representation in base $(m_1,\dots,m_s)$, we cannot take benefit of precomputing every
$\beta^i \bmod m_j$ since each precomputations will be used only once per moduli. However, this is not the case when reducing many integers with the same moduli. Indeed, in such case,  the
precomputations can serve for every integer you need to reduce. In this section, we provide a novel way to simultaneously convert to and from RNS that take advantage of such precomputations and
introduce matrix multiplication into the complexity. In particular, we will achieve a complexity of $\bigO(rs^{\omega-1}\M(t))$ bit operations where $\omega$ is the exponent in the complexity of
matrix multiplication, best value being $\omega=2.3729$ \cite{LeGall:2014}


\subsubsection{Conversions to RNS}


The conversion to RNS is split up into three phases. 
\begin{itemize}
\item  First, one need to compute all the $|\beta^i|_{m_j}=\beta^i \bmod m_j$ for $1\leq i,j\leq s$ and group them into the following matrix 
\[
B=
\begin{bmatrix}
1 & |\beta|_{m_1} & |\beta^2|_{m_1} & \dots & |\beta^{s-1}|_{m_1} \\
\vdots & \vdots & \vdots & \ddots  & \vdots \\
1 & |\beta|_{m_s} & |\beta^2|_{m_s} & \dots & |\beta^{s-1}|_{m_s} \\
\end{bmatrix}\in \Mat_{s \times s} (\Z) .
\]
\item Then, use this matrix to pseudo-reduce the $r$ integers $[a_1,\dots, a_r]$ modulo the $m_i$'s. Here, pseudo-reduction means that we reduce integers of bitsize $\bigO(s)$ to bitsize $\bigO(\log
  s)$. For this purpose, we write the expansion in base $\beta$ of 
\[ a_i = \sum_{j = 0}^{s - 1} \alpha_{i,j} \beta^j \mbox{\quad for } 1 \leqslant i \leqslant r\]
and group them into the following matrix

\[
C=
\begin{bmatrix}
\alpha_{1,0} & \alpha_{2,0} & \alpha_{3,0} & \dots & \alpha_{r,0} \\
\vdots & \vdots & \vdots & \ddots  & \vdots \\
\alpha_{1,s-1} & \alpha_{2,s-1} & \alpha_{3,s-1} & \dots & \alpha_{r,s-1} \\
\end{bmatrix} \in \Mat_{s \times r} (\Z) 
\]
From the definition of matrices $B$ and $C$ we have 
\begin{equation}
(BC)_{i,j} \equiv a_j \bmod m_i
\end{equation}
and 
\begin{equation}
0 \leq (BC)_{i,j} < s m_i\beta < s \beta^2.
\end{equation}
This shows that the matrix product $BC$ encode all the pseudo-reduction of the $a_i$'s with the $m_j$'s.
\item The last step consist in reducing the $i$-th row of the matrix product $BC$ modulo $m_i$ for $1 \leqslant i \leqslant r$.
\end{itemize}


The complexity of this approach is $\bigO(\M\M(s,s,r)\M(t))$ bit operations.
Indeed, the dominant cost is achieved for the matrix product $BC$ where each entries of both matrices are bounded by $\beta=2^t$. The construction of matrix $B$ only costs $\bigO(s^2\M(t))$ bit operations
while the last step to reduce the entries in $BC$ costs $\bigO(sr\M(t))$ bit operations since by assumption $s<\beta$.

Assuming $s\leq r$, this gives a complexity of $\bigO(rs^{\omega-1}\M(t))$. This boils down to $\bigO(r^{2(\omega-2)}s^2\M(t))$ when $s>r$.
 
% Using one matrix
% multiplication, we are able to reduce the computation of pseudo-reductions of
% $a_i \bmod m_\ell $ to the reductions of $\beta^j \rem m_\ell$ for
% $0 \leqslant j < n$ and $1 \leqslant \ell \leqslant s$ and linear algebra. These
% latter reductions do not depend on the number $a$ to reduce and are precomputed
% once and for all. Then we turn the pseudo-reductions into reductions in
% negligible time.

% \paragraph*{Simultaneous pseudo-reductions}

% The first thing to do is to write the expansion in base $\beta$ of
% \[ a_i = \sum_{j = 0}^{n - 1} c_{i,j} \beta^j \]
% for $1 \leqslant i \leqslant r$. Let's precompute the values
% $r_{j, \ell} \assign \beta^j \rem m_\ell$ for $0 \leqslant j < n$ and
% $1 \leqslant \ell \leqslant s$.  This costs
% $\bigO \left( s^2 \I (\log (\beta)) \right)$ by computing them incrementally.

% Let $a_{i, \ell} \assign \sum_{j = 0}^{n - 1} c_{i, j} r_{j, \ell}$ so that 
% $a_{i, \ell} \equiv a_i \bmod m_{\ell}$. The value $a_{i, \ell} $ is bounded by
% $n \beta^2$, when $a_i$ was of size $\beta^n$. We say that $a_{i, \ell}$ is a
% pseudo-reduction of $a_i$ modulo $p_{\ell}$.
% The values $a_{i, \ell}$ can be computed by linear algebra :
% $(a_{i, \ell}) \in \Mat_{r \times s} (\Z)$ is the product of
% $(c_{i, j}) \in \Mat_{r \times n} (\Z)$ and
% $(r_{j, \ell}) \in \Mat_{n \times s} (\Z)$.

% This product costs $\bigO( \M\M(r,n,s) \I(\log \beta))$ since
% $n \leqslant \beta$ by assumption (**see Introduction**). In the most common
% case $n = \bigO(s)$ which corresponds to $a_j < m_1 \cdots m_s$, the complexity
% reduces to $\left( rs^{\omega - 1} \I (\log (\beta)) \right)$.

% \paragraph*{Simultaneous reductions}Now the cost of computing the remainder
% $(a \rem m)$ when $a < n \beta^2$ and $m < \beta$ is
% $\bigO \left( \I (\log (\beta)) \right)$ using again the initial simplifying
% assumption. Therefore, our final step to compute our simultaneous reductions
% costs $\bigO \left( r s \I (\log (\beta)) \right)$.

\subsubsection{Conversions from RNS}

\paragraph*{Simultaneous pseudo-reconstructions}

Recall that we denote $M=\Pi_{i=1}^sm_i$ and $M_i=(M/m_i) \rem m_i$ for
$1 \leqslant i \leqslant s$.  Let
$l_i \assign \sum_{j = 1}^s a_{i, j} M_j [M_j^{- 1}]_{m_j}$ so that
$a_i = l_i \bmod M$ (see Formula \eqref{eq:cra}) and $l_i < M \beta$. We say
that $l_i$ are pseudo-reconstructions of $(a_{i, \ell})$ modulo
$m_1, \ldots, m_s$.

Let us precompute $M_{\ell} [M_{\ell}^{- 1}]_{m_{\ell}}$ for all
$1 \leqslant \ell \leqslant s$. If we have precomputed the $M_\ell$ before
(\emph{e.g.} during the reduction step), then we can compute
$[M_{\ell}]_{m_{\ell}}$ from $M_\ell$ in time $\bigO(s \I(\log \beta))$, then
$[M_{\ell}^{-1}]_{m_{\ell}}$ in time $\bigO(\I(\log \beta) \log \log \beta)$ and
finally $M_{\ell} [M_{\ell}^{- 1}]_{m_{\ell}}$ in time
$\bigO(s \I(\log \beta))$. When $\beta$ is a constant, this comes down to
$\bigO(s)$ bit operations.


Now from the expansion of $M_{\ell} [M_{\ell}^{- 1}]_{m_{\ell}}$
in base $\beta$
\[
M_j [M_j^{- 1}]_{m_j} = \sum_{k = 0}^{s - 1} e_{j, k} \beta^k.
\]
we perform the computation of $l_i$ using linear algebra using
\[ 
l_i = \sum_{j = 1}^s a_{i, j} M_j [M_j^{- 1}]_{m_j} = \sum_{j = 1}^s
a_{i, j} \sum_{k = 0}^{s - 1} e_{j, k} \beta^k = \sum_{k = 0}^{s - 1} \left(
  \sum_{j = 1}^s a_{i, j} \cdot e_{j, k} \right) \beta^k . 
\]

Let $(d_{i, j}) \in \Mat_{r \times s}(\Z)$ be the product of the matrices
$(a_{i, j}) \in \Mat_{r \times s}(\Z)$ and
$(e_{j, k}) \in \Mat_{s \times s}(\Z)$. Then
$l_i = \sum_{k = 0}^{s - 1} d_{i, j} \beta^k$.
Note that $(d_{i, j})$ are not the exact coefficients of the $\beta$-expansion
of $l_i$.
But we can still compute the $l_i$s in time $\bigO(r s \I(\log \beta))$
since $d_{i, j} \leqslant s \beta^2$ and $s \leqslant \beta$ by assumption.
The matrix product to compute $(d_{i, j})$ can be done in bit complexity
$\bigO \left( rs^{\omega - 1} \I (\log (\beta)) \right)$.

\paragraph{Simultaneous reconstructions}The final step of the reconstruction
consists in reducing $l_i$ modulo $M$. This step is relatively cheap since $l_i$
is almost reduced.  Using $s < \beta$ and the modular reduction complexity
results (**see Introduction**) when $\log (a / p) \ll \Theta (\log (p))$ in our
case $l_i =\bigO (s \beta^{s + 2}) = \bigO(\beta^{s + 2})$ and
$M =\bigO (\beta^s)$, the last reduction step costs
$\bigO \left( s \I (\log \beta) \right)$ per $l_i$. Thus a total cost of
$\bigO (r s \I (\log \beta))$.

\section{Matrix multiplication with multi-precision integer coefficients}

\subsection{Matrix multiplication modulo composite primes}

Of course, this ideal case of application of matrix multiplication using RNS is
for matrices over $\Z/M\Z$ with $M=m_1 \cdots m_s$ and $m_i < \beta$. Then, the
multiplication of two matrices $A \in \Mat_{n,k}(\Z/M\Z)$,
$B \in \Mat_{k,m}(\Z/M\Z)$ is done classically by converting them to RNS,
multiply each of the reductions and convert from RNS. The cost is
$$\bigO( \left[\M\M(n,k,m) s + (nk + km) s^{2} \right] \I(\log \beta))$$ using the
naive approach for RNS,
$$\bigO( \left[\M\M(n,k,m) s + (nk + km) s^{\omega-1} \right] \I(\log \beta))$$
using linear algebra approach and
$$\bigO( \left[\M\M(n,k,m) s + (nk + km) \I(s) \log s \right] \I(\log \beta))$$
using the quasi-optimal approach.

\subsection{Integer matrix multiplication}\label{ssec:imm}

In this section, the matrices $A \in \Mat_{n,k}(\Z)$, $B \in \Mat_{k,m}(\Z)$
have entries of absolute value less than $\beta^n$.  Therefore, their product
$C \assign A \cdot B$ have entries of absolute value less than $k \beta^{2n}$.

We will compute $C$ modulo $M=m_1 \cdots m_s$ such that $2 k \beta^{2n} < M$ and
$m_i < \beta$. Therefore, the entries of $C$ coincide with their unique residue
modulo $M$ whose absolute value is less than $\frac{M-1}{2}$.

Since we make the simplifying assumption that $k < \beta$ and
assuming there exists enough primes slightly smaller than $\beta$, we will
necessarily have $s = \Theta(n)$. Therefore, we can state the costs:
$$\bigO( \left[\M\M(n,k,m) n + (nk + km) n^{2} \right] \I(\log \beta))$$ using the
naive approach for RNS,
$$\bigO( \left[\M\M(n,k,m) n + (nk + km) n^{\omega-1} \right] \I(\log \beta))$$
using linear algebra approach and
$$\bigO( \left[\M\M(n,k,m) n + (nk + km) \I(n) \log n \right] \I(\log \beta))$$
using the quasi-optimal approach.

**Aarrrg ! We have the same notation : $n$ is both the matrix size and its bit size**

\subsection{Matrix multiplication modulo any prime}

In the case of matrix multiplication over $\Z/M\Z$ when $M$ do not decompose
into the product of several small primes, we actually perform an integer matrix
multiplication and reduce the result modulo $M$. The costs remain the same than
in Section~\ref{ssec:imm} with the bit size $n \assign \log_\beta M$.  Note that
each final reduction costs $\bigO(\I(n) \I(\log \beta))$ (see
Section~\ref{sssec:naivetoRNS}), which is negligible.



\section{Implementation}
{\color{red}
All the codes are available in the FFLAS\_FFPACK Library (see https://github.com/linbox-team/fflas-ffpack).
\begin{itemize}
\item The simultaneous reduction/reconstruction's code is available in the file:\\ \texttt{fflas-ffpack/fflas-ffpack/field/rns-double.inl}
\item The matrix multiplication code is available in the file: \\\texttt{fflas-ffpack/fflas-ffpack/fflas/fflas\_fgemm/fgemm\_classical\_mp.inl}
\end{itemize}
}
\subsection{Reduction to word-size matrix multiplication}
how to store intermediate matrices (Kronecker matrix) in order to optimize cache
efficiency (transpose is delayed to the matrix multiplication). We always want
the Kronecker matrix to be stored $\beta$-adic major.  How to choose $\beta$ and
the $m_i$ to guarantee the result to fit into a word-size register.

\subsection{Kronecker substitution}
$\beta$ is chosen to be $2^{16}$ in order to do not require any arithmetic
operation.  Can we do better with a larger beta ? What is the compromise between
$\beta$ and the $m_i$ ?

Reconstruction from the almost $\beta$-adic to GMP integer using a splitting of
the $beta$-adic expansion into four GMP integer. ? Can we do better do reduce the
result mod $\prod m_i$.

\subsubsection{From integer to $\beta$-adic}
using gmp structure we can do this for free. only cast mpz data to uint16.

\subsection{Linear storage for multi-modular matrix}
explication lda, rda , RNSMajor ou MatrixMajor
\section{Benchmarks}


\section{TODO}
1) parallelisme 2) algo hybride (rapide/la)

%\newpage
%\tableofcontents

% Partie 1 : Introduction

\bibliographystyle{acmtrans} \bibliography{biblio}
\end{document}
  
